\section{Abstract}

Il progetto propone lo sviluppo di una base dati per la gestione efficiente dell'illuminazione pubblica in Italia, in risposta alla crisi internazionale determinata dal conflitto Russia-Ucraina, che ha portato a un aumento esponenziale del prezzo del gas e alla necessità di ridurre le emissioni di CO2 e i consumi energetici. La base dati mira a fornire una soluzione efficace per ottimizzare l'uso delle risorse energetiche e ridurre i costi associati all'illuminazione pubblica, considerando le misure di risparmio energetico promosse dall'Unione Europea e dall'Italia.
L’obiettivo è sviluppare un’applicazione {\it{web responsive}} in grado di monitorare e di eseguire le azioni sopra menzionate sul sistema di illuminazione pubblico. L'applicazione deve soddisfare molti requisiti e avere molte operazioni che possono essere migliorate e coadiuvate da un sistema informativo. Molte operazioni infatti sono ripetitive, lavorano su grosse quantità di dati o sono richieste ad orari anomali. Ad esempio in presenza di crepuscolo o luce lunare particolarmente intensa, per i quali sia possibile ridurre l’intensità luminosa mantenendo un giusto grado di illuminazione, non sempre è possibile avere tutto il personale disponibile per gestire le richieste.
Un aspetto fondamentale della gestione dei sistemi e delle applicazioni moderne è il salvataggio in maniera intelligente dei log, in quanto offre numerosi benefici come il miglioramento delle prestazioni del sistema, la risoluzione rapida dei problemi, il rilevamento delle minacce alla sicurezza e la generazione di informazioni utili per l'analisi e il miglioramento continuo. È quindi 