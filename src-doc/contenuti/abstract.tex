\section{Abstract}

La crisi internazionale determinata dal conflitto Russia-Ucraina sta cambiando gli assetti geopolitici. Oltre ai notevoli costi sociali ed economici, ripropone a livello globale la centralità dei temi energetici e ambientali.
Il risparmio delle risorse del Pianeta e in particolare delle fonti energetiche è entrato con forza nell'agenda politica dell’Unione Europea. Fra gli impatti più evidenti, spicca la crescita esponenziale del prezzo del gas, risorsa ancora largamente utilizzata come materia prima per la produzione di energia elettrica. Mentre in sede di Commissione Europea è in corso un acceso dibattito sul tetto al prezzo del gas, in Italia si susseguono misure atte al risparmio energetico (dal “Piano nazionale per il risparmio del gas naturale” ai successivi decreti per arginare il cosiddetto “caro-bollette” per imprese, famiglie, etc.).
Accanto a tali criticità, figurano il cambiamento climatico e surriscaldamento globale che rappresentano un’ulteriore emergenza, nota da tempo e ancora oggi molto trascurata: gli altissimi livelli di emissioni di CO2 e la gestione dei cosiddetti “gas serra” stanno modificando irreversibilmente il clima, il Pianeta e l’eredità che lasceremo alle generazioni future. Molti movimenti dal basso, fra cui anche Fridays for Future, si stanno facendo strada in questo ambito, cercando di sensibilizzare le multinazionali e i governi europei. Il mix energetico (gas, nucleare, rinnovabile, carbone) necessario a soddisfare il fabbisogno italiano è ad oggi fortemente polarizzato su gas e carbone, che inevitabilmente porta all’aumento delle emissioni di gas serra e CO2.
Per far fronte al rincaro delle bollette energetiche, molti Comuni italiani stanno annunciando il taglio dell’illuminazione pubblica, che necessita di una quantità considerevole di energia elettrica. Stando ai dati, il consumo annuale di energia elettrica per illuminazione pubblica in Italia nel 2016 è stato di circa 6.000 GWh, mentre il consumo pro-capite è stato di circa 100 kWh, ovvero il doppio della media europea di 51 kWh (dati 2017). Il costo complessivo per pubblica illuminazione in Italia è stato di 1,7 miliardi di euro, ovvero circa 28,70 € pro-capite.