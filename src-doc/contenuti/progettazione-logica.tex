\chapter{Progettazione logica}

\section{Eliminazione delle generalizzazioni}

\paragraph{Misuratore}

Per evitare di accorpare le entità Sensore e Lampione nell'entità padre Misuratore, creando così dei campi NULL, è stato deciso di risolvere la generalizzazione mantenendo le tre entità inserendo le relazioni tra le entità figlie e l'entità padre.

\section{Modifiche, aggiunte e chiarimenti alle chiavi}

Tutte le chiavi primarie soino definite utilizzando le chiavi della progettazione concettuale, tranne nei seguenti casi.

\paragraph{Sensore} Diventando il Sensore un'entità a sé stante, vengono aggiunte una chiave primaria ed una chiave esterna che indica il Misuratore a cui fa riferimento.

\paragraph{Lampione} Diventando il Lampione un'entità a sé stante, vengono aggiunte una chiave primaria ed una chiave esterna che indica il Misuratore a cui fa riferimento.

\section{Schema concettuale ristrutturato - Schema Logico}

%% immagine

\section{Descrizione schema relazionale}

La chiave primaria è indicata in \textbf{grassetto}, le chiavi esterne sono indicate con la \underline{sottolineatura}.

\textit{Utente}(\textbf{username}, nome, cognome, email, password, \underline{ruolo}) \\
\textit{Ruolo}(\textbf{id}, nome) \\
\textit{Area}(\textbf{id}, nome, autoMode, lvlInf, lvlSup) \\
\textit{Misuratore}(\textbf{id}, \underline{idArea}, tipo, latitudine, longitudine) \\
\textit{Sensore}(\textbf{id}, \underline{idMisuratore}, raggio) \\
\textit{Lampione}(\textbf{id}, \underline{idMisuratore}, luminosita)

\section{Vincoli di integrità referenziali}

\textbf{Utente}.ruolo -> \textit{Ruolo}.id \\
\textbf{Sensore}.idMisuratore -> \textit{Misuratore}.id \\
\textbf{Lampione}.idMisuratore -> \textit{Misuratore}.id

\section{Check e constraint}

\paragraph{Area} In \textbf{Area} è attivato un check che controilla che il livello inferiore sia sempre minore del livello superiore.