\section{Analisi delle entità}

\textbf{Se non specificato l'attributo è NOT NULL}

%UTENTE
\begin{center}
    \begin{tabularx}{\textwidth}{|l|l|l|X|}
        \hline
        \rowcolor{gray!30}
        \multicolumn{4}{|c|}{\textbf{UTENTE}}\\
        \hline
        username & VARCHAR(20) & Identifica univocamente un utente nel sistema & Chiave\\
        \hline
        nome & VARCHAR(50) & \multicolumn{2}{l|}{Il nome dell'utente} \\
        \hline
        cognome & VARCHAR(50) & \multicolumn{2}{l|}{Il nome dell'imbarcazione} \\
        \hline
        email & VARCHAR(50) & \multicolumn{2}{l|}{La quantità di posti letto dell'imbarcazione} \\
        \hline
        password & VARCHAR(20) & \multicolumn{2}{l|}{Nome dello stato di cui batte bandiera l'imbarcazione} \\
        \hline
        ruolo & INTEGER & Identifica il ruolo dell'utente & Chiave esterna: Ruolo(id)\\
        \hline
    \end{tabularx}
\end{center}

%RUOLO
\begin{center}
    \begin{tabularx}{\textwidth}{|l|l|l|X|}
        \hline
        \rowcolor{gray!30}
        \multicolumn{4}{|c|}{\textbf{RUOLO}}\\
        \hline
        id & INTEGER  & Identifica univocamente un ruolo che può essere ricoperto dall'utente & Chiave\\
        \hline
        nome & VARCHAR(50) & \multicolumn{2}{l|}{Il nome del ruolo} \\
        \hline
    \end{tabularx}
\end{center}

%AREA
\begin{center}
    \begin{tabularx}{\textwidth}{|l|l|l|X|}
        \hline
        \rowcolor{gray!30}
        \multicolumn{4}{|c|}{\textbf{AREA}}\\
        \hline
        id & INTEGER & Identifica univocamente un'area all'interno del sistema & Chiave\\
        \hline
        nome & VARCHAR(50) & \multicolumn{2}{l|}{Il nome dell'area} \\
        \hline
        autoMode & BOOLEAN & \multicolumn{2}{l|}{Specifica se l'area viene gestita in modalità automatica o no (manuale)} \\
        \hline
        lvlInf & VARCHAR(50) & \multicolumn{2}{l|}{In area gestita in modalità automatica, questo è il livello di luminosità} \\ & & \multicolumn{2}{l|}{a cui vengono posti i lampioni se non vengono rilevate persone all'interno dell'area} \\
        \hline
        lvlSup & VARCHAR(20) & \multicolumn{2}{l|}{In area gestita in modalità automatica, questo è il livello di luminosità} \\ & & \multicolumn{2}{l|}{a cui vengono posti i lampioni se vengono rilevate persone all'interno dell'area} \\
        \hline
    \end{tabularx}
\end{center}

%MISURATORE
\begin{center}
    \begin{tabularx}{\textwidth}{|l|l|l|X|}
        \hline
        \rowcolor{gray!30}
        \multicolumn{4}{|c|}{\textbf{MISURATORE}}\\
        \hline
        id & SERIAL & Identifica univocamente un misuratore del livello di luminosità & Chiave\\
        \hline
        tipo & VARCHAR(10) & \multicolumn{2}{l|}{La tipologia del misuratore} \\
        \hline
        latitudine & REAL & \multicolumn{2}{l|}{Latitudine delle coordinate geografiche in cui viene posto il misuratore} \\
        \hline
        longitudine & REAL & \multicolumn{2}{l|}{Longitudine delle coordinate geografiche in cui viene posto il misuratore} \\
        \hline
        idArea & INTEGER & Identifica l'area a cui fa riferimento il misuratore & Chiave esterna: Area(id)\\
        \hline
    \end{tabularx}
\end{center}

%SENSORE
\begin{center}
    \begin{tabularx}{\textwidth}{|l|l|X}
        \hline
        \rowcolor{gray!30}
        \multicolumn{3}{|c|}{\textbf{SENSORE}}\\
        \hline
        raggio & INTEGER & Il raggio d'azione entro il quale il sensore è in grado di rilevare persone \\
        \hline
    \end{tabularx}
\end{center}

%LAMPIONE
\begin{center}
    \begin{tabularx}{\textwidth}{|l|l|X|}
        \hline
        \rowcolor{gray!30}
        \multicolumn{3}{|c|}{\textbf{LAMPIONE}}\\
        \hline
        luminosita & INTEGER & Livello di luminosità in cui si trova il lampione \\
        \hline
    \end{tabularx}
\end{center}