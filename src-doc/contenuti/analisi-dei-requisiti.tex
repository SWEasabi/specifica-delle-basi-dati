\chapter{Analisi dei Requisiti}

\section{Descrizione testuale}
Si vuole realizzare una base di dati per il progetto di Ingegneria del Software \textit{Lumos Minima}, che dovrà gestire tutti i dati relativi agli utenti e al sistema di illuminazione.

Le funzioni desiderate prevedono:
\begin{itemize}
    \item autenticazione degli utenti;
    \item operazioni CRUD per misuratori (lampioni/sensori) ed aree;
    \item regolazione dell'intensità luminosa;
    \item modifica dello stato di illuminazione di un'area da manuale ad automatico e viceversa;
    \item sistema di logging per le modifiche all'intensità luminosa dei lampioni;
    \item sistema di logging per le rilevazioni dei sensori.
\end{itemize}

Si prevede che il sistema di illuminazione abbia un numero variabile di aree, sensori e sensori, in quanto è un servizio in continua espansione e modifica.

Gli utenti del sistema possono essere di due tipi: gestore o manutentore, in base alla loro mansione. Tuttavia si prevede la possibilità di aggiungere altri tipi di utenti in futuro.

Dell'area si vuole tenere traccia della modalità di illuminazione, che può essere automatico o manuale. In particolare, nel caso sia in modalità automatica, si vuole sapere i due livelli di illuminosità (superiore e inferiore) che può avere l'area, in base alla presenza di persone nelle vicinanze.
In ogni area sono presenti un numero variabile di misuratori. Di essi si vuole tenere traccia della posizione e della sua tipologia, che può essere sensore o lampione. Nel primo caso si vuole sapere il suo raggio d'azione, mentre nel secondo caso si vuole conoscere la sua intensità luminosa.

Ogni volta che viene modificata l'intensità luminosa di un lampione, si vuole tenere traccia del nuovo valore di essa e dell'istante temporale dell'evento.
Per il sensore invece, si vuole tenere traccia di quando vengono rilevate persone o meno.